%\documentclass[manuscript]{biometrika}
\documentclass[lineno]{biometrika}

\usepackage{amsmath}
%\usepackage{ntheorem}

%\usepackage{subcaption}
%\usepackage{multirow}

\usepackage{times}
\usepackage{bm}
\usepackage{natbib}
\usepackage{color}

\setlength{\textfloatsep}{10pt plus 1.0pt minus 2.0pt}
\setlength{\floatsep}{12.0pt plus 2.0pt minus 5.0pt}
\setlength{\intextsep}{12.0pt plus 2.0pt minus 5.0pt}
\setlength{\belowcaptionskip}{-2pt}
 
%
% Math commands by Thomas Minka 
%
% Revised by Jyotishka Datta & Brandon Willard
% Acknowledgement: JD received this from Prof. Alan Qi.
%


%% Special shortcuts for us
\def\Polya{P{\'o}lya}
\def\CS{Cauchy-Schl\"omilch}
\def\PG{P{\'o}lya-Gamma}

%\setlength{\textfloatsep}{10pt plus 1.0pt minus 2.0pt}
%\setlength{\floatsep}{12.0pt plus 2.0pt minus 5.0pt}
%\setlength{\intextsep}{12.0pt plus 2.0pt minus 5.0pt}
%\setlength{\belowcaptionskip}{-2pt}

%\def\distrib{\mathrel{\ooalign{%
%  \raisebox{0.75\height}{{\small{ind}}}\cr\hidewidth$\sim$\hidewidth\cr}}}
%  
\newcommand{\var}{{\rm var}}
\newcommand{\Tr}{^{\rm T}}
\newcommand{\rmlog}{\rm log}
\newcommand{\vtrans}[2]{{#1}^{(#2)}}
\newcommand{\kron}{\otimes}
\newcommand{\schur}[2]{({#1} | {#2})}
\newcommand{\schurdet}[2]{\left| ({#1} | {#2}) \right|}
\newcommand{\had}{\circ}
\newcommand{\diag}{{\rm diag}}
\newcommand{\invdiag}{\diag^{-1}}
\newcommand{\rank}{{\rm rank}}
% careful: ``null'' is already a latex command
\newcommand{\nullsp}{{\rm null}}
\newcommand{\tr}{{\rm tr}}
\renewcommand{\vec}{{\rm vec}}
\newcommand{\vech}{{\rm vech}}
\renewcommand{\det}[1]{\left| #1 \right|}
\newcommand{\pdet}[1]{\left| #1 \right|_{+}}
\newcommand{\pinv}[1]{#1^{+}}
\newcommand{\erf}{{\rm erf}}
\newcommand{\hypergeom}[2]{{}_{#1}F_{#2}}

% boldface characters
\renewcommand{\a}{{\bf a}}
\renewcommand{\b}{{\bf b}}
\renewcommand{\c}{{\bf c}}
\renewcommand{\d}{{\rm d}}  % for derivatives
\newcommand{\e}{{\rm e}} % for exponentials
\newcommand{\f}{{\bf f}}
\newcommand{\g}{{\bf g}}
\newcommand{\h}{{\bf h}}
%\newcommand{\k}{{\bf k}}
% in Latex2e this must be renewcommand
\renewcommand{\k}{{\bf k}}
\newcommand{\m}{{\bf m}}
\newcommand{\n}{{\bf n}}
%\renewcommand{\o}{{\bf o}}
\newcommand{\p}{{\bf p}}
%\newcommand{\q}{{\bf q}}
\renewcommand{\r}{{\bf r}}
\newcommand{\s}{{\bf s}}
\renewcommand{\t}{{\bf t}}
\renewcommand{\u}{{\bf u}}
\renewcommand{\v}{{\bf v}}
\newcommand{\w}{{\bf w}}
%\newcommand{\x}{{\bf x}}
\newcommand{\y}{{\bf y}}
%\newcommand{\z}{{\bf z}}
\newcommand{\A}{{\bf A}}
\newcommand{\B}{{\bf B}}
%\newcommand{\C}{{\bf C}}
\newcommand{\D}{{\bf D}}
\newcommand{\E}{{\bf E}}
\newcommand{\F}{{\bf F}}
%\newcommand{\G}{{\bf G}}
\renewcommand{\H}{{\bf H}}
\newcommand{\I}{{\bf I}}
\newcommand{\J}{{\bf J}}
\newcommand{\K}{{\bf K}}
\renewcommand{\L}{{\bf L}}
\newcommand{\M}{{\bf M}}
\newcommand{\Nor}{{\cal N}}  % for normal density
%\newcommand{\N}{{\bf N}}
\renewcommand{\O}{{\bf O}}
\renewcommand{\P}{{\bf P}}
\newcommand{\Q}{{\bf Q}}
\newcommand{\R}{{\bf R}}
%\renewcommand{\S}{{\bf S}}
%\newcommand{\T}{{\bf T}}
%\newcommand{\U}{{\bf U}}
\newcommand{\V}{{\bf V}}
\newcommand{\W}{{\bf W}}
\newcommand{\X}{{\bf X}}
\newcommand{\Y}{{\bf Y}}
\newcommand{\Z}{{\bf Z}}

% this is for latex 2.09
% unfortunately, the result is slanted - use Latex2e instead
%\newcommand{\bfLambda}{\mbox{\boldmath$\Lambda$}}
% this is for Latex2e
\newcommand{\bfLambda}{\boldsymbol{\Lambda}}

% Yuan Qi's boldsymbol
\newcommand{\bsigma}{\boldsymbol{\sigma}}
\newcommand{\balpha}{\boldsymbol{\alpha}}
\newcommand{\bpsi}{\boldsymbol{\psi}}
\newcommand{\bphi}{\boldsymbol{\phi}}
\newcommand{\bbeta}{\boldsymbol{\beta}}
%\newcommand{\Beta}{\boldsymbol{\eta}}
\newcommand{\btau}{\boldsymbol{\tau}}
\newcommand{\bvarphi}{\boldsymbol{\varphi}}
\newcommand{\bzeta}{\boldsymbol{\zeta}}
\newcommand{\bnabla}{\boldsymbol{\nabla}}
\newcommand{\blambda}{\boldsymbol{\lambda}}
\newcommand{\bLambda}{\mathbf{\Lambda}}

\newcommand{\btheta}{\boldsymbol{\theta}}
\newcommand{\bpi}{\boldsymbol{\pi}}
\newcommand{\bPi}{\boldsymbol{\Pi}}
\newcommand{\bxi}{\boldsymbol{\xi}}
\newcommand{\bSigma}{\boldsymbol{\Sigma}}

\newcommand{\bgamma}{\mathbf{\gamma}}
\newcommand{\bGamma}{\mathbf{\Gamma}}

\newcommand{\bmu}{\boldsymbol{\mu}}
\newcommand{\bnu}{\boldsymbol{\nu}}
\newcommand{\bPsi}{\mathbf{\Psi}}
\newcommand{\bepsilon}{\boldsymbol{\epsilon}}
\newcommand{\bOmega}{\boldsymbol{\Omega}}

\newcommand{\1}{{\bf 1}}
\newcommand{\0}{{\bf 0}}

%\newcommand{\comment}[1]{}

\newcommand{\bs}{\backslash}
\newcommand{\ben}{\begin{enumerate}}
\newcommand{\een}{\end{enumerate}}
\newcommand{\beq}{\begin{equation}}
\newcommand{\eeq}{\end{equation}}

 \newcommand{\notS}{{\backslash S}}
 \newcommand{\nots}{{\backslash s}}
 \newcommand{\noti}{{\backslash i}}
 \newcommand{\notj}{{\backslash j}}
 \newcommand{\nott}{\backslash t}
 \newcommand{\notone}{{\backslash 1}}
 \newcommand{\nottp}{\backslash t+1}
% \newcommand{\notz}{\backslash z}

\newcommand{\notk}{{^{\backslash k}}}
%\newcommand{\noti}{{^{\backslash i}}}
\newcommand{\notij}{{^{\backslash i,j}}}
\newcommand{\notg}{{^{\backslash g}}}
\newcommand{\wnoti}{{_{\w}^{\backslash i}}}
\newcommand{\wnotg}{{_{\w}^{\backslash g}}}
\newcommand{\vnotij}{{_{\v}^{\backslash i,j}}}
\newcommand{\vnotg}{{_{\v}^{\backslash g}}}
\newcommand{\half}{\frac{1}{2}}
\newcommand{\quart}{\frac{1}{4}}
\newcommand{\msgb}{m_{t \leftarrow t+1}}
\newcommand{\msgf}{m_{t \rightarrow t+1}}
\newcommand{\msgfp}{m_{t-1 \rightarrow t}}

\newcommand{\proj}[1]{{\rm proj}\negmedspace\left[#1\right]}
\newcommand{\argmin}{\operatornamewithlimits{argmin}}
\newcommand{\argmax}{\operatornamewithlimits{argmax}}

\newcommand{\dif}{\mathrm{d}}
\newcommand{\abs}[1]{\lvert#1\rvert}
\newcommand{\norm}[1]{\lVert#1\rVert}
\newcommand{\vectornorm}[1]{\left|\left|#1\right|\right|}

\newcommand{\Norm}{\mathcal{N}}
\newcommand{\bx}{{\bf x}}
\newcommand{\ba}{{\bf a}}
\newcommand{\bb}{{\bf b}}
\newcommand{\bc}{{\bf c}}
\newcommand{\bd}{{\bf d}}
\newcommand{\bX}{{\bf X}}
\newcommand{\by}{{\bf y}}
\newcommand{\IG}{\mathcal{IG}}
\newcommand{\dd}[2]{\frac{\partial #1}{\partial #2}}
\newcommand{\lhat}[1][i]{\hat\lambda_{#1}^{-1(g)}}
\newcommand{\what}[1][j]{\hat\omega_{#1}^{-1(g)}}
\newcommand{\bone}{{\bf 1}}
\newcommand{\Li}{\hat\Lambda^{-1(g)}}
\newcommand{\Oi}{\hat\Omega^{-1(g)}}
\newcommand{\iid}{\stackrel{\mathrm{iid}}{\sim}}
\newcommand{\iidp}{\stackrel{\mathrm{P}}{=}}
\newcommand{\iidd}{\stackrel{\mathrm{D}}{=}}
\newcommand{\defeq}{\operatorname{:=}}

% the last {} is a hack for double subscript errors
\newcommand{\estHsp}{\ensuremath{{\hat{\theta}}_{HS+}}{}}
\newcommand{\estHs}{\ensuremath{{\hat{\theta}}_{HS}}{}}
\newcommand{\estJs}{\ensuremath{{\hat{\theta}}_{JS}}{}}
\newcommand{\MSE}{\mathrm{MSE}}
%
% Meijer-G additions
%
%\DeclarePairedDelimiterX\MeijerM[3]{\lparen}{\rparen}%
%{\begin{smallmatrix}#1 \\ #2\end{smallmatrix}\delimsize\vert\,#3}
%
%\newcommand\MeijerG[8][]{%
%  G^{\,#2,#3}_{#4,#5}\MeijerM[#1]{#6}{#7}{#8}}
%
%\WithSuffix\newcommand\MeijerG*[7]{G^{\,#1,#2}_{#3,#4}\MeijerM*{#5}{#6}{#7}}
%% end Meijer-G
%
%%
%% Generalized Hypergeometric Function (pFq)
%%
%\DeclarePairedDelimiterX\pFqM[3]{\lparen}{\rparen}%
%{\begin{smallmatrix}#1 \\ #2\end{smallmatrix}\delimsize\vert\,#3}
%
%\newcommand\pFq[6][]{%
%  {}_{#2}F_{#3}\pFqM[#1]{#4}{#5}{#6}}

%\WithSuffix\newcommand\pFq*[5]{{}_{#1}F_{#2}\pFqM*{#3}{#4}{#5}}
% end pFq

%\newtheorem{theorem}{Theorem}
%\numberwithin{theorem}{section}
%\newtheorem{Proof}{Proof}
%\newtheorem{Def}{Definition}
%\numberwithin{Def}{section}
%\newtheorem{remark}{Remark}
%\numberwithin{remark}{section}
%\newtheorem{Qes}{Question}
%\newtheorem{proposition}{Proposition}
%\numberwithin{proposition}{section}
%\newtheorem{lemma}{Lemma}
%\numberwithin{lemma}{section}
%\newtheorem{Cor}{Corollary}
%\numberwithin{Cor}{section}
%\newtheorem{Exa}{Example}
%\newtheorem{Eq}{Equation}
%\newtheorem{assn}{Assumption}
%\newtheorem{result}[theorem]{Result}


%\usepackage{booktabs,array}
%\def\Midrule{\midrule[\heavyrulewidth]}
%\newcount\rowc
%
%\makeatletter
%\def\ttabular{%
%\hbox\bgroup
%\let\\\cr
%\def\rulea{\ifnum\rowc=\@ne \hrule height 1.0pt \fi}
%\def\ruleb{
%\ifnum\rowc=1\hrule height 1.0pt  \else
%\ifnum\rowc=3\hrule height 0.0pt%\heavyrulewidth 
%\ifnum\rowc= 3  \hrule height 0.5pt \else%\heavyrulewidth 
%\ifnum\rowc= 5  \hrule height 0.5pt \else%\heavyrulewidth 
%\ifnum\rowc= 7  \hrule height 0.5pt \else%\heavyrulewidth 
%\ifnum\rowc= 9  \hrule height 0.5pt \else%\heavyrulewidth 
%\ifnum\rowc= 11  \hrule height 0.5pt %\heavyrulewidth 
  %\else \hrule height 0pt%\lightrulewidth
%\fi\fi\fi\fi\fi\fi}
%\valign\bgroup
%\global\rowc\@ne
%\rulea
%\hbox to 7em{\strut \hfill##\hfill}%
%\ruleb
%&&%
%\global\advance\rowc\@ne
%\hbox to 7em{\strut\hfill##\hfill}%
%\ruleb
%\cr}
%\def\endttabular{%
%\crcr\egroup\egroup}



\begin{document}

\jname{Biometrika}
\jyear{2016}
\jvol{103}
\jnum{1}
%\doi{10.1093/biomet/asm023}
\accessdate{Advance Access publication on 31 July 2016}
%\copyrightinfo{\Copyright\ 2016 Biometrika Trust \goodbreak {\em Printed in Great Britain}}

%\received{April 2016}
%\revised{September 2016}

\markboth{Bhadra, Datta, Polson \and Willard}{Global-local mixtures}

\title{Global-local mixtures}

\author{Anindya Bhadra}
\affil{Department of Statistics, Purdue University, \\ 250 N. University St.,
West Lafayette, Indiana 47907, U.S.A. \email{bhadra@purdue.edu.}}
\author{Jyotishka Datta}
\affil{Department of Mathematical Sciences, University of Arkansas, \\
Fayetteville, Arkansas 72701, U.S.A. \email{jd033@uark.edu}}
\author{Nicholas G. Polson \and Brandon Willard }
\affil{Booth School of Business, The University of Chicago,  5807 S. Woodlawn
Ave., Chicago, Illinois 60637, U.S.A. \email{ngp@chicagobooth.edu \and
brandonwillard@gmail.com}}


\maketitle
\begin{abstract}
  Global-local mixtures are derived from the \CS{} and Liouville integral transformation identities. We characterize well-known normal-scale mixture
  distributions including the Laplace or lasso, logit and quantile as well as new global-local mixtures. We also apply our methodology to convolutions that
  commonly arise in Bayesian inference. Finally, we conclude with a conjecture concerning bridge and uniform correlation mixtures. 
\end{abstract}

\begin{keywords}
Bayes regularization; Cauchy; Convolution; Global-local mixture; Lasso;  Logistic; Quantile; Stable law. % 3 to 8 kwds, alphabetically. Deleted Scale mixture;
\end{keywords}

\section{Introduction}
Many statistical problems involve regularization penalties derived from
global-local mixture distributions \citep{polson_data_2011,
hans2011comment,bhadra2015horseshoe+}. A global-local mixture density, denoted
by $p(x_1, \ldots, x_p)$, takes the form 
\begin{equation*}
  p(x_1, \ldots, x_p) = \int_{0}^{\infty}\prod_{i=1}^{p} p(x_i \mid \tau) 
  p(\tau) d\tau, 
\end{equation*}
where $p(x_i \mid \tau) = \int_{0}^{\infty} p(x_i \mid \lambda_i, \tau) p(\lambda_i \mid \tau) d\lambda_i$ is a local mixture and $p(x_1, \ldots, x_p)$ is a global mixture over $\tau \sim p(\tau)$. There is great interest in analytically calculating $p(x_i \mid \tau)$, and the associated regularization penalty $\phi(x_i, \tau) = -\log p(x_i \mid \tau)$.  Convolution mixtures of the form $p(x_i \mid \tau) = \int p(x_i - \lambda_i) p(\lambda_i) d \lambda_i$ are also of interest. We show how the \CS{} and Liouville transformations can
be used to derive closed-form global-local mixtures.  We start by stating two key integral identities: 
the \CS{} transformation 
\begin{equation}
  \int_0^\infty f \left\{ ( a x - b x^{-1} )^2 \right\} d x 
  = \frac{1}{2a} \int_0^\infty f(y^2) d y
  \;, 
  \label{eq:identity}
\end{equation}
and the Liouville transformation
\begin{equation}
  \int_{0}^{\infty} f\left(ax + \frac{b}{x} \right) x^{-1/2}dx = a^{-1/2} \int_{0}^{\infty} f\left\{ 2 (ab)^{1/2} + y \right\} y^{-1/2} dy, \quad a, b >0
  \;. 
  \label{eq:liouville}
\end{equation}
See \citet{boros2006irresistible}, \citet{baker2008probabilistic} and \citet{jones_generating_2014} for further discussion. Identity \eqref{eq:identity} follows from the simple transformation $t = b/(a x)$ as
\begin{equation*}
  I = \int_{0}^{\infty} f \left\{(ax - b/x)^2 \right\} dx = \int_{0}^{\infty} f \left\{(at - b/t)^2 \right\} \frac{b}{a t^2} dt.
\end{equation*}
Adding the two terms in the last equality yields $2 I = \int_{0}^{\infty} f \left\{(at - b/t)^2 \right\} \left\{ 1+{b}/({a t^2}) \right\} dt$ and transforming $y = b/t - at$ gives $dy = -a \{1+{b}/({a t^2})\} dt$, which yields $I = (2a)^{-1} \int_{0}^{\infty} f(y^2) dy$, as required. A useful generalization of the \CS{} transformation is
\begin{equation}
  \int_0^\infty f\left[ \{x-s(x)\}^2 \right] dx =  \int_0^\infty f( y^2 ) dy, \label{eq:gen}
\end{equation}
where $s(x)=s^{-1}(x)$ is a self-inverse function such as $s(x) = b/x$ or $s(x) = -a^{-1}\log\{1-\exp(a x)\}$. The proof for the Liouville transformation identity follows in a
similar manner, and is omitted for the sake of brevity. These identities can be used to construct new global-local mixture distributions. 
Let $f(x) = 2g\{ t(x) \}$ and let $t(x)$ be of the form $x-s(x)$, where $s : \Re^+ \to \Re^+$ is a self-inverse, onto and monotone decreasing function. Together with the \CS{} transformation, we have a rather surprising way to represent the resulting $g\{t(x)\}$ as a global-local scale mixture. 

\citet{jones_generating_2014} shows that only a few choices of $t(x)$ leads to fully tractable formulae for its inverse $t^{-1}= \Pi$ and the integral 
$\Pi(y) = \int_{-\infty}^{y} \pi(\omega) d\omega$. Two special choices are the $t$-distribution with 2 degrees of freedom and the logistic. 
\begin{align*}
\Pi_{T}(y) = (1/2)\{ y+(4b+y^2)^{1/2}\}, \Pi_T^{-1}(x) = t_T(x) = x - b/x, \quad b >0,\\
\Pi_{L}(y) = a^{-1} \log(1+e^{ay}), \Pi_L^{-1}(x) = t_L(x) = a^{-1} \log(e^{ax}-1), \quad a>0.
\end{align*}
%We can use these results to generate new probability distributions with
%different choices of simple baseline function $g(\cdot)$ and derive new scale
%mixture representations that are useful in Bayesian global-local modeling. 

Now, the integral identity in \eqref{eq:identity} shows that if $f(x), \;x\geq 0$ is a density function, so is $g(x) = 2a f(|ax-b/x|), \; x > 0$.  The functions $f$ and $g$ are called mother and daughter density functions, respectively.  %\citet{chaubey2010reciprocal} provide a one-to-one correspondence between $f$ and $g$. 

Apart from simplifying proofs involving global-local mixtures, the \CS{} and Liouville transformations can generate new distributions via scale transformations. These transformations can take the form $f(x) = 2 g\{ t(x) \}$ for certain $f(x)$ under suitable conditions. For example, given a density $f(x)$ we can create a new global-local
scale family, $f(a x - b/x)$, by effectively reallocating its probability mass. A particularly useful tool for generating univariate and multivariate random
variables is Khintchine's theorem.  
Khintchine's theorem states that any random variable $X$ with a unimodal, univariate distribution and a mode at zero can be written as a product 
$X = Z U$, where $U \sim \UnifRV(0,1)$ and $Z$ has the density function $f_Z(z) = -z f^{\prime}_{X}(z), z \in \Re$. \citet{bryson1982constructing}, and subsequently \citet{jones2012khintchine}, discuss how Khintchine's theorem allows one to construct both univariate and multivariate densities, even with special dependence structure. \cite{jones_generating_2014} develops an extended Khintchine's theorem that further allows one to generate random variables with unimodal densities of the
form $2 g\{t(x)\}$. 

%%% maybe add one or two lines
%Furthermore if $f(x)$ is the pdf of a symmetric real-valued random variable
%$X$, the daughter pdf $g(x) = f(x-1/x), x>0$ is an R-symmetric density, and
%vice-versa, there exists a symmetric density $f(x) = g(x+\sqrt{1+x^2})$ for
%every R-symmetric density $g(x)$. Furthermore, $f(\cdot)$ is unimodal if and
%only if $g(\cdot)$ is unimodal. \cite{chaubey2010reciprocal} provides a few
%examples of generating R-symmetric densities $g$ starting from well-known
%symmetric densities $f$. 
%
%Apart from providing simple proofs for well-known normal scale mixture
%representations, \CS and Liouville transformations can generate new
%distributions by a suitable scale transformation $f(x) = 2g\{ t(x) \}$ of a
%simple baseline function $f$ under some conditions, and for $t(x) = x - 1/x$,
%we get R-symmetric densities on $\Re^+$ from well-known symmetric densities
%and vice-versa, and given a density $f(x)$ we can create a new global-local
%scale family $f(ax-bx^{-1})$, by reallocating its probability mass. 
%
%A particularly useful tool for generating univariate and multivariate random
%variables is Khintchine's theorem for unimodality of univariate distributions
%\citep{bryson1982constructing}. Khintchine's theorem states that any random
%variable $X$ with a mode at zero can be written as a product $X = ZU$, where
%$U \sim U(0,1)$ and $Z$ has the density function $f_Z(z) = -zf_{X}'(z)$.
%\citet{bryson1982constructing} and successively \citet{jones2012khintchine}
%discuss how Khintchine's theorem allows us to construct both univariate and
%multivariate distributions, even with special dependence
%structure.\cite{jones2014generating} develops an extended Khintchine's theorem
%that further lets one generate random variable with unimodal density of the
%form $2g(t(x))$. 

%The rest of the paper is organized as follows: \S\ref{sec:gls_mixes} derives scale mixture results for the Lasso, quantile and logistic regression, \S\ref{sec:convolutions} for convolutions of densities via mixtures and finally \S\ref{sec:discussion} concludes with two open problems. 

\section{Global-local Scale Mixtures}
\label{sec:gls_mixes}
\subsection{Lasso as a normal scale mixture}
The lasso penalty arises as a Laplace global-local mixture
\citep{andrews_scale_1974}.  A simple transformation proof follows using \CS{}
with $f(x) = e^{-x}$.  Starting with the normal integral identity, 
$\int_{0}^{\infty} f(y^2) dy = \int_0^\infty e^{-y^2} dy = \pi^{1/2}/2 $, we
obtain
\[
\int_0^\infty e^{-(a x)^2 - (b/x)^2} d x = \int_0^{\infty} 
\exp\left\{-a b \left(\frac{a}{b} x^2 + \frac{b}{a} x^{-2} \right)\right\} dx 
= \frac{\pi^{1/2}}{2a} e^{-2 a b}
\quad a,b \in \Re.
\]
Substituting $t = (a/b)^{1/2} x$ and $c = ab$ yields the Laplace or Lasso penalty as
\begin{align*}
  \int_0^\infty e^{- c (t - t^{-1})^2} dt 
  &= \half (\pi/c)^{1/2} 
  \Rightarrow \int_0^\infty e^{- c (t^2 + t^{-2})} dt 
  = \half (\pi/c)^{1/2} e^{-2c}
  \;. 
  %\label{eq:andrews}
\end{align*}
The Laplace density can be viewed as a transformed normal, via $y = t - t^{-1}$.

\begin{proposition}
The usual identity for the lasso also follows from \citet{levy1940certains} as
\begin{equation}
  \int_{0}^{\infty} \frac{a}{(2 \pi)^{1/2} t^{3/2}} e^{-{a^2}/({2 t})} e^{-\lambda t} dt = e^{-a (2 \lambda)^{1/2} } \;.\label{eq:levy}
\end{equation}
For $a = 1$, and $\theta = (2 \lambda)^{1/2}$, this can be written as 
\begin{equation}
  E [\exp\{-\theta^2/(2G)\}] = \exp(-\theta),\quad G \sim \GammaRV(1/2, 1/2) 
  \label{eq:gamma}
\end{equation}
\end{proposition}

\begin{proof}
First substitute $t^{-1} = x^2$, which makes the left hand side in
\eqref{eq:levy} equal to 
\[
  \int_{0}^{\infty} \frac{a}{(2 \pi)^{1/2} t^{3/2}} e^{-{a^2}/({2 t})} e^{-\lambda t} dt = \left(\frac{2}{\pi}\right)^{1/2}ae^{-a (2 \lambda)^{1/2}} 
  \int_0^{\infty} e^{-({2}^{-1/2} ax - \lambda x^{-1})^2} dx = e^{-a (2 \lambda)^{1/2}}
  \;.
\]
The last step follows from \CS{} formula.  The second relationship \eqref{eq:gamma} follows by fixing $a = 1$, $\theta = (2\lambda)^{1/2}$ and
substituting $t = x^{-1}$
\[
\int_{0}^{\infty} \frac{a}{(2 \pi)^{1/2} t^{3/2}} 
e^{-{a^2}/({2 t})} e^{-\lambda t} dt 
= \frac{1}{(2 \pi)^{1/2}} \int_{0}^{\infty} e^{-{\theta^2}/({2x})} 
x^{-1/2} e^{-x/2} dx.
\]
The left hand side can be identified as 
$E\left\{e^{-\theta^2 / (2 G) } \right\}$ for 
$G \sim \GammaRV(1/2, 1/2)$. 
\end{proof}

%% hyperbolic-GIG \citep{barndorff1977infinite}
\subsection{Logit and quantile as global-local mixtures}
Logistic modeling can be viewed within the global-local mixture framework via the \PG{} distribution \citep{polson_bayesian_2013}. This leads to efficient Markov chain Monte Carlo algorithms for inference. 
\begin{proposition}
The two key marginal distributions for the hyperbolic generalized inverse Gaussian \citep{barndorff1982normal} and \PG{} mixtures are
\begin{align}
 \frac{\alpha^2 - \kappa^2}{2\alpha} e^{-\alpha|x-\mu| + \kappa (x-\mu)} &= \int_0^{\infty} \phi(x \mid \mu + \kappa \lambda, \lambda) p_{\mathrm{GIG}}\left\{ \lambda \mid 1,0, (\alpha^2 - \kappa^2)^{1/2}\right\} d\lambda, \; \alpha \geq \kappa \geq 0, \label{eq:GIG}\\
\frac{1}{B(\alpha,\kappa)} \frac{e^{\alpha (x-\mu)}}{(1+e^{x-\mu})^{\alpha + \kappa}}&= \int_0^{\infty} \phi(x \mid \mu + \kappa \lambda, \lambda)p_{\mathrm{Polya}}(\lambda \mid \alpha,\kappa)  d\lambda\;, \label{eq:polya}
\end{align}
where $\phi(\mu + \kappa \lambda, \lambda)$ denotes the normal density function with mean $(\mu + \kappa \lambda)$ and variance $\lambda$.  The functions $p_{\mathrm{GIG}}$ and $p_{\mathrm{Polya}}$ are the corresponding local mixture densities for the generalized inverse Gaussian and the \PG{}, respectively. The logit and quantile identities can be derived using \CS{} identity. 
\end{proposition}
\begin{proof}
Let $f(x) = e^{-x^2/2}$, $a = \alpha$ and $b = |x-\phi|$ in \eqref{eq:identity}. Then,
\[
(2/\pi)^{1/2} \int_{0}^{\infty} 
\exp\left\{-\half \left(\alpha y - \frac{|x-\mu|}{y} \right)^2 \right\} dy 
= \frac{1}{\alpha} (2\pi)^{-1/2} \int_0^{\infty} e^{-\half y^2} dy 
= \frac{1}{\alpha}
\;.
\]
Let $\nu = y^2$. Rearranging the constant terms yields
\[
\frac{1}{\alpha} e^{-\alpha|x-\mu|} = \frac{1}{(2 \pi \nu)^{1/2}} \int_{0}^{\infty} \exp\left[-\left\{ \frac{(x-\mu)^2}{2\nu} + \frac{\alpha^2}{2} \nu \right\} \right]
d\nu
\;.
\]
Multiplying by $2^{-1}(\alpha^2-\kappa^2) e^{\kappa(x-\mu)}$ and completing the square yields
\begin{equation*}
  \frac{\alpha^2-\kappa^2}{2\alpha} \exp\left\{-\alpha|x-\mu| + \kappa(x-\mu)\right\} 
  = \int_0^{\infty} \phi(x \mid \mu + \kappa \nu, \nu) 
  \frac{\alpha^2-\kappa^2}{2} \exp\left(-\frac{\alpha^2-\kappa^2}{2} \nu \right) d \nu. 
\end{equation*}
The mixing distribution is exponential with rate parameter $(\alpha^2-\kappa^2)/2$, a special case of the generalized inverse Gaussian distribution introduced by Etienne Halphen circa 1941
\citep{seshadri1997halphen}.  The density with parameters $(\lambda, \delta, \gamma)$ 
has the form 
\begin{equation*}
  p_{\mathrm{GIG}}(x \mid \lambda, \delta, \gamma) = \frac{(\gamma/\delta)^{\lambda}}{2 K_{\lambda}(\delta \gamma)} x^{\lambda-1} 
  \exp\left\{ -\half (\delta^2 x^{-1} + \gamma^2 x )\right\}, \quad x, \lambda, \delta > 0,\;  p \in \Re
  \;,
\end{equation*}
where $K_{\lambda}$ is the modified Bessel function of the second kind.  The Liouville formula can be used to show that the above is a valid probability density
function.  When $\delta$ or $\gamma$ is zero, the normalizing constant takes the limiting values given by $K_{\lambda}(u) \asymp \Gamma(|\lambda|) 2^{|\lambda|-1} u^{|\lambda|}$ 
for $\lambda > 0$.  If $\delta=0$, the generalized inverse Gaussian is identical to a gamma distribution:
\[
p_{\mathrm{GIG}}(x \mid \lambda, \delta = 0 , \gamma) 
= \frac{\alpha^{\lambda}}{\Gamma(\lambda)} x^{\lambda-1} \exp(-\alpha x), 
\quad x > 0,\; \alpha = \gamma^2 / 2.
\]
%\end{proof} 
%
%%%%%%%%%%%
%\begin{proof}
%\noindent \textbf{Proof of \eqref{eq:polya}} \\
We now present a simple proof for the \PG{} mixture in \eqref{eq:polya}. 
First, write $\kappa$ for $a-b/2$: 
\begin{equation}
  \frac{(e^{\psi})^a}{(1+e^{\psi})^b} = 2^{-b} e^{\kappa \omega} 
  \int_0^{\infty} e^{-\omega \psi^2/2} p(\omega) d\omega
  \;, 
  \label{eq:pg}
\end{equation}
where $\omega \sim \operatorname{PG}(b,0)$, a \PG{} random variable with density
$$
p(\omega \mid b, 0) = \frac{2^{b-1}}{\Gamma(b)} 
\sum_{n=0}^{\infty} (-1)^n \frac{\Gamma(n+b)}{\Gamma(n+1)} 
\frac{2n + b}{(2 \pi)^{1/2} \omega^{3/2}} 
\exp\left\{-\frac{(2 n + b)^2}{8 \omega} \right\}.
$$
The logit function corresponds to $a=0,b=1$ in \eqref{eq:pg}. \CS{} identity yields
\begin{equation}
  \frac{1}{1+e^{\psi}} = \half e^{- \psi/2} \int_0^{\infty} e^{-(\psi^2\omega)/2} p(\omega) d\omega,
  \quad p(\omega) = \sum_{n=0}^{\infty} (-1)^n \frac{2n+1}{ (2 \pi \omega^3)^{1/2}} 
  e^{-(2n+1)^2/(8 \omega)}
  \label{eq:logit}
  \;.
\end{equation}
To show \eqref{eq:logit}, write the right-hand side interchanging the integral and summation:
\begin{align*}
  %I & = \half e^{-\psi/2} \int_0^{\infty} e^{-\frac{\psi^2}{2} \omega}
  %\sum_{n=0}^{\infty} (-1)^n \frac{2n+1}{\sqrt{(2\pi)} \omega^{3/2}}
  %e^{-\frac{(2n+1)^2}{8\omega}}d \omega \\
  I & = \half e^{-\psi/2} \sum_{n=0}^{\infty} (-1)^n \frac{2n+1}{(2 \pi)^{1/2}} 
  \int_0^{\infty} 
  \exp\left[-\left\{ \frac{\psi^2}{2} \omega + \frac{(2n+1)^2}{8 \omega} \right\} \right] \frac{1}{\omega^{3/2}} d\omega
  \;. 
\end{align*}
Using the change of variable $\omega = t^{-2}$ gives
\begin{align*}
  % I & = e^{-\psi/2}  \sum_{n=0}^{\infty} (-1)^n \frac{(2n+1)}{\sqrt{(2\pi)}}
  % \int_{0}^{\infty} e^{-\left( \frac{\psi^2}{2t^2} + \frac{(2n+1)^2 t^2}{8}
  % \right)} d t \\
  I & = \sum_{n=0}^{\infty} (-1)^n e^{-(n+1)\psi} 
  \frac{2n + 1}{(2 \pi)^{1/2}} 
  \left( \int_{0}^{\infty} 
    \exp\left[-\half \left\{ \frac{(2n+1)t}{2} - \frac{\psi}{t}\right\}^2 \right] dt 
  \right)
  \;.
\end{align*}
Applying the \CS{} identity to the inner integral yields 
\[
\int_{0}^{\infty} 
\exp\left[-\half \left\{ \frac{(2n+1)t}{2} - \frac{\psi}{t}\right \}^2 \right] dt 
= \int_0^{\infty} \frac{e^{-y^2/2}}{2n+1} dy= \frac{(2\pi)^{1/2}}{2n+1}
\;,
\]
which implies $I = \sum_{n=0}^{\infty} (-1)^n \exp\{-(n+1) \psi\} = \{1+\exp(\psi)\}^{-1}$. 
%An alternative proof using Laplace transformation is provided in
%\cite{polson2013bayesian}.  
\end{proof}

\begin{remark}
When $\alpha = \kappa$, we have the limiting result $(\alpha^2-\kappa^2)^{-1} p_{\mathrm{GIG}}\{1,0, (\alpha^2-\kappa^2)^{1/2} \} = 1,$
or equivalently in terms of densities, with a marginal improper uniform prior, $p(\lambda) = 1$,
\begin{equation}
  \int_{0}^{\infty} \phi(b \mid -a\lambda, c\lambda) d\lambda = a^{-1} \exp\left\{-2 \max(ab/c,0)\right\}
  \;. 
  \label{eq:svm}
\end{equation}
This pseudo-likelihood represents support vector machines as a global-local mixture. The identity for quantile regression, which is a limiting case of the above identities by applying Fatou-Lebesgue theorem, is the following: 
\[
c^{-1}\exp\{ 2c^{-1} \rho_q(b) \}= \int_{0}^{\infty} \phi( b \mid \lambda - 2\tau \lambda, c \lambda){\rm e}^{-2\tau(1-\tau)\lambda} d\lambda, \quad c, \tau > 0,
\]
where $\rho_q(b) = \rvert b \lvert / 2 + (q-1/2) b$ is the check-loss function \citep{polson_data_2013}.
\end{remark}

\citet{polson_data_2011} derive this as a direct consequence of the lasso identity $\int_0^{\infty} p/(2 \pi \lambda)^{1/2} \exp\left\{-\left(p^2 \lambda+q^2 \lambda^{-1}\right)/2\right\} d\lambda = e^{-\lvert pq \rvert}$. Applying the Liouville identity yields
\[
\int_{0}^{\infty} f\left(ax + \frac{b}{x} \right) x^{-1/2} dx = a^{-1/2} \int_{0}^{\infty} f\left\{ 2 (ab)^{1/2} + y \right\} y^{-1/2} dy, \quad a, b > 0.
\]
Setting $f(x) = e^{-x}$, $a = p^2/2$, and $b = q^2/2$ we get
%\[
\begin{align*}
  \int_0^{\infty} \frac{e^{-(p^2 \lambda + q^2 \lambda^{-1})/2}}{\lambda^{1/2}} d\lambda
  & = \frac{2^{1/2}}{p} \int_0^{\infty} e^{-|pq| + y} y^{-1/2} d y \\
  & = \frac{2^{1/2} e^{-|pq|}}{p} \int_0^{\infty} e^{-y} y^{-1/2} d y 
  = \frac{(2\pi)^{1/2} e^{-|pq|}}{p}
  \;.
\end{align*}
%\]

\citet{hans2011comment} shows that the elastic-net regression can be recast as a
global-local mixture with a mixing density belonging to the orthant-normal
family of distributions.  The orthant-normal prior on a single regression
coefficient, $\beta$, given hyper-parameters $\lambda_1$ and $\lambda_2$, 
has a density function with the following form:
\begin{equation}
  p(\beta \mid \lambda_1, \lambda_2)  = 
  \begin{cases} 
   \phi(\beta \mid \frac{\lambda_1}{2\lambda_2}, \frac{\sigma^2}{\lambda_2}) 
   / 2\Phi\left(-\frac{\lambda_1}{2\sigma \lambda_2^{1/2} }\right), & \quad \beta < 0, 
   \\
   \phi(\beta \mid \frac{-\lambda_1}{2\lambda_2}, \frac{\sigma^2}{\lambda_2}) / 
   2\Phi\left(-\frac{\lambda_1}{2\sigma \lambda_2^{1/2} }\right), & \quad \beta \geq 0.
  \end{cases} 
  \;
  \label{eq:hans}
\end{equation}

%% Commented out 08/23
%\citet{hans2011comment} then proves the following theorem that comes as a
%corollary of \eqref{eq:svm}: \begin{theorem} Under the orthant normal prior on
%$\beta$ and a prior on $\lambda_2$ with the density \begin{equation*}
%p(\lambda_2 \mid \lambda_1, \sigma^2) =
%\frac{\lambda_1^2}{2\sigma^2\lambda_2^2} \Phi \left( \frac{-\lambda_1}{2\sigma
%\sqrt{\lambda_2}} \right), \lambda_2 > 0 \end{equation*} the induced marginal
%prior is $(\beta \mid \lambda_1, \sigma^2) \sim
%\mathrm{DE}(\lambda_1/\sigma^2)$, where $\mathrm{DE(\lambda)}$ denotes the
%double-exponential distribution.  \end{theorem} This result comes as a
%corollary of \eqref{eq:svm}.  the limiting result stated in the last remark,
%i.e. it follows from the identity $$ \int_{0}^{\infty} \phi(b \mid -a\lambda,
%c\lambda) d\lambda = a^{-1} exp(-2 \max(ab/c,0)).  $$ \subsection{Stable Laws}
%The symmetric stable distribution is defined by its characteristic function
%$\phi(t) = \exp\{ -|t|^{\alpha} \}$, with exponent $\alpha \in (0,2]$. The
%symmetric stable distribution also admits a normal scale mixture
%representation as where the mixing density is given by: $$ f(v) = \half
%s_{\alpha/2} \left( \frac{v}{2} \right), v > 0 $$ where $s_{\alpha/2}$ is the
%density of the positive stable distribution with index $\alpha / 2$ (Feller,
%1971). This result can also shown to be a consequence of the integral identity
%\eqref{eq:identity}.  \textit{I am still working on the proof of it.}

%\section{Transformations of Scale Distribution} \textcolor{red}{Needs more
%Statistical motivation. We can use the $Y = aX-bX^{-1}$ tranformation examples
%here, or, the Lasso scale mixture example by using $y = t - t^{-1}$ on $f(x) =
%e^{-x}$. Another way would be to move the $t_2$ and logistic examples to start
%the section.}

%\begin{remark} \subsection{Symmetric and R-symmetric distributions}

%For absolutely continuous random variables supported on the positive half-line
%$\Re^{+}$, the density function $f(\cdot)$ has reciprocal symmetry
%(R-symmetry) if $f(\theta y) = f(\theta / y)$ for all $y > 0$ and some $\theta
%>0$ (Mudholkar and Wang, 2007). The most well-known example of this duality is
%perhaps the normal density as $f$ that gives rise to the root reciprocal
%inverse Gaussian, abbreviated as RRIG, distribution, with density given by: $$
%g(x) = \sqrt{\frac{2\lambda}{\pi}} \exp \left\{ - \frac{\lambda}{2} \left( x -
%\frac{1}{x} \right)^2 \right\}, x >0.  $$ Once again, the \CS transformation
%$y = x - x^{-1}$ shows this is a density. 

\section{Convolution mixtures}
\label{sec:convolutions}

Another interesting area of application is convolution mixtures and marginal densities for location-scale mixture problems. We show that the Cauchy
convolution \citep{pillai2015unexpected} and inverse-gamma convolution can be derived similarly \citep{polson_halfcauchy_2012}. \citet{bhadra_default_2016} shows that the regularly varying tails of half-Cauchy priors work well for low-dimensional functions of normal
vector mean, where flat priors give poorly calibrated inference. 
%% Inverse-Gamma appears in \citep{arnold2009some}.
\begin{lemma}
  Let $X_i \sim \CauchyRV(0,1)$ $(i = 1, 2)$ be Cauchy distributed random variates, then $Z = w_1 X_1 + w_2 X_2 \sim \CauchyRV( 0, w_1 + w_2).$ where $w_1,w_2 > 0$.
\end{lemma}
\begin{lemma}
  Let $X_i \sim \InvGaussRV(\alpha t_i, \alpha t_i^2)$ $(i = 1, 2)$, then $Z = X_1 + X_2 \sim \InvGaussRV\{\alpha (t_1 + t_2), \alpha (t_1^2+t_2^2)\},$ where $\alpha, t_1, t_2 \geq 0$, and $\InvGaussRV(\alpha t, \alpha t^2)$ is an inverse-Gaussian random variable with density
\[
    f(x) = \frac{t \alpha^{1/2} e^t}{(2 \pi)^{1/2} x^{3/2}} 
    \exp\left( -\frac{\alpha t^2}{2x} - \frac{x}{2\alpha} \right), \quad x \geq 0.
\]
\end{lemma}

Both of these results follow from straightforward applications of the \CS{} 
transformation. We give a proof for the Cauchy convolution identity below.

\begin{proof}
Exploiting symmetry and the Lagrange identity $(a^2 + b^2)(c^2 + d^2) = (ac+bd)^2 + (ad-bc)^2,$ leads to the convolution density
\begin{align*} 
  f_Z(z) &= 2 \int_{0}^{\infty} 
  \frac{1}{ \pi w_1 (1+ x^2/w_1^2)} \frac{1}{\pi w_2 \{1+ (z-x)^2 / w_2^2 \} } dx
  \\
  & = \frac{2}{\pi^2 w_1 w_2} \int_{0}^{\infty} 
  \frac{1}{\{1+ w_1^{-1} w_2^{-1} x (z-x) \}^2 + \{w_2^{-1}z - (w_1^{-1}+ w_2^{-1}) x \}^2 } dx.
\end{align*}
Transforming $x$ to $x + w_2^{-1}z (w_1^{-1} + w_2^{-1})^{-1}$ and letting $a = 1 + z^2(w_1+w_2)^{-2}$, $b =(w_1 w_2)^{-1}$, $c = z (w_2-w_1) \{(w_1+w_2) w_1 w_2\}^{-1}$, $d = z (w_2-w_1)\{(w_1+w_2) w_1 w_2\}^{-1}$ gives
\begin{align*}
  f_Z(z) &= \frac{2}{\pi^2 w_1 w_2} \int_{0}^{\infty} 
  \left[ 
    \left\{ 1 + \frac{z^2}{(w_1+w_2)^2} - \frac{x^2}{w_1w_2} + 
      xz \frac{w_2-w_1}{(w_1+w_2) w_1 w_2} \right\}^2 + 
    x^2 \left(\frac{w_1 + w_2}{w_1w_2} \right)^2 
  \right]^{-1} dx 
  \\
  &= \frac{2}{\pi^2 w_1 w_2} \int_{0}^{\infty} 
  \frac{dx}{\left( a - b x^2 + cx \right)^2 + x^2 d^2} 
  = \frac{2}{\pi^2 w_1 w_2} \int_{0}^{\infty} 
  \frac{dx/x^2}{\left(a/x - bx + c \right)^2 + d^2 }.
\end{align*}
If we let $y = x^{-1}$ and apply the \CS{} transformation, we arrive at 
\[
  f_Z(z) = \frac{2}{\pi w_1 w_2} \int_{0}^{\infty} \frac{dy}{2a (y^2 + d^2)} 
  = \frac{1}{\pi w_1 w_2} \frac{1}{ad}= \frac{1}{\pi (w_1+w_2)} \frac{1}{1+ z^2/(w_1+w_2)^2}.
%& = \frac{1}{\pi w_1 w_2} \frac{1}{1+ z^2/(w_1+w_2)^2} \frac{w_1 w_2}{w_1 + w_2} = \frac{1}{\pi (w_1+w_2)} \frac{1}{1+ z^2/(w_1+w_2)^2}
\]
A simple induction argument proves that the sum of any number of independent
Cauchy random variates is also another Cauchy.
\end{proof}

One can also use the characteristic function of $X \sim \CauchyRV(\mu, \sigma)$, $\psi_X(t) = \exp(it \mu - |t| \sigma^2)$, and the relation $\psi_{X+Y}(t) = \psi_X(t) \psi_Y(t)$ to derive the result in just one step. For $X = \sum_{i=1}^{p} \omega_i C_i$ and $C_i \sim \CauchyRV(0,1)$, when $\sum_{i=1}^{p} \omega_i = 1$ we have 
$\phi_X(t) = \exp\left(-\sum_{i=1}^{p}\omega_i |t|\right) = \exp(-|t|) = \phi_C(t),$ where $C \sim \CauchyRV(0, 1)$. 

The most general result in this category is due to \cite{pillai2015unexpected},
who they showed the following: 
Let $(X_1,\ldots,X_m)$ and $(Y_1, \ldots, Y_m)$ be independent and
identically distributed $\NormRV(0, \Sigma)$ for an arbitrary
positive definite matrix $\Sigma$, then 
$Z = \sum_{j=1}^{m} w_j X_j / Y_j \sim \CauchyRV(0, 1)$, 
as long as $(w_1, \ldots, w_m)$ is independent of $(X, Y)$,
$w_j \geq 0\ (j = 1, \ldots, m)$ and $\sum_{j=1}^{m} w_j = 1$. 

\section{Discussion}
\label{sec:discussion}

The \CS{} and Liouville transformations not only guarantee an simple normalizing constant for $f(\cdot)$, it also establishes the wide class of unimodal densities as global-local scale mixtures. Global-local scale mixtures that are conditionally Gaussian hold a special place in statistical modeling and can be rapidly fit using an expectation-maximization algorithm, as
pointed out by \cite{polson_data_2013}. \cite{palmer_amica:_2011} provides a similar tool for modeling multivariate dependence by writing general non-Gaussian multivariate densities as multivariate Gaussian scale mixtures. %Our future goal is to extend the \CS{} transformation to express the wide multivariate Gaussian scale mixture models as global-local mixtures that also facilitate easy computation.

We end our paper with conjectures that two other remarkable identities arise as corollaries of such transformation identities. The first one is a recent result by \cite{zhang2014uniform} that proves a uniform correlation mixture of a bivariate Gaussian density with unit variance is a function of the maximum norm: 
\begin{equation}
  \int_{-1}^{1} \frac{1}{4 \pi (1-\rho^2)^{1/2} } 
  \exp\left\{ - \frac{x_1^2 + x_2^2 - 2 \rho x_1 x_2}{2 (1-\rho^2)} \right\} d\rho = 
  \half \left\{1- \Phi(\vectornorm{x}_{\infty})\right\} 
  \;, 
  \label{eq:bivar}
\end{equation}
where $\Phi(\cdot)$ is the standard normal distribution function and $\vectornorm{x}_{\infty} = \max\{ x_1, x_2\}$. The bivariate density on the
right side of \eqref{eq:bivar} was introduced by \citet{bryson1982constructing} as uniform mixtures of a chi random variate with 3 degrees of freedom, but the representation as a uniform correlation mixture is a new find.  We make a few remarks connected to the Erdelyi's integral identity, which is key to the proof of the uniform correlation mixture of \eqref{eq:bivar}. 
%revise
\begin{lemma}
Erdelyi's identity, defined by
\begin{equation}
  \int_{1/2}^{\infty} \frac{e^{-x^2 z}}{4 \pi z 	(2z-1)^{1/2}} dz = \half \left\{1-\Phi(x)\right\}, \quad x \geq 0, \label{eq:erdelyi}
\end{equation}
follows from the Laplace transformation $(1+u)^{-1} = \int_0^{\infty} \exp\{-v(1+u)\} dv$. 
\end{lemma}

\begin{proof}
  Apply the transform $u = 2z-1$ to the left hand side of \eqref{eq:erdelyi}, denoted by $I$, to obtain 
  \[
  I = \int_{0}^{\infty} \frac{e^{-x^2/\{2(1+u)\}}}{4 \pi {u}^{1/2} (1+u)} du \;.
  \]
  Using the Laplace transformation $(1+u)^{-1} = \int_0^{\infty} e^{-v(1+u)} dv$ yields
  \begin{align*}
    I &= \int_{0}^{\infty} \frac{e^{-x^2/\{2(1+u)\}}}{4 \pi {u}^{1/2}} 
    \int_0^{\infty} e^{-v(1+u)} dv du 
    = \int_{v= 0}^{\infty} \int_{u=0}^{\infty} 
    \frac{e^{-({x^2}/{2} + v)(1+u)}}{4 \pi {u}^{1/2}} dv du
    \\
    &= \int_{v= 0}^{\infty} \frac{1}{4\pi} e^{-({x^2}/{2} + v)} 
    \int_{u=0}^{\infty} u^{-1/2} e^{-({x^2}/{2} + v) u} du dv 
    = \int_{v= 0}^{\infty} \frac{e^{-(x^2 + 2v)/2}}{2 (2\pi)^{1/2}} 
    \frac{1}{(x^2+ 2v)^{1/2} } dv
    %&= \int_{v= 0}^{\infty} \frac{\e^{-(\frac{x^2}{2} + v)}}{4\pi} \frac{\pi^{1/2}}{(\frac{x^2}{2} + v)^{1/2}} dv 
    %= \int_{v= 0}^{\infty} \frac{1}{2 (2\pi)^{1/2}} e^{-\half(x^2 + 2v)} \frac{1}{(x^2+ 2v)^{1/2}} dv
    \;,
  \end{align*}
  and letting $z^2 = x^2 + 2v$ we get 
  \begin{equation*}
    I = \half \int_{z = |x|}^{\infty} \frac{1}{(2\pi)^{1/2}} e^{-z^2/2} dz 
    = \half \left\{1 - \Phi(|x|)\right\} \;.
  \end{equation*} 
  %\proofSymbol
  % hack for automatic qed symbol
\end{proof}

%%% [[ Commented out as Amdeberhan is not published yet and Biometrika won't let us cite ArXiv articles ~ JD. ]] %%%%

%Erdelyi's identity in \eqref{eq:erdelyi} follows from (9.2) of
%\cite{amdeberhan_cauchy-schlomilch_2010}: 
%\[
%\int_0^{\infty} \frac{e^{-\mu^2 (x^2 + \beta^2)}}{x^2+\beta^2} dx = \frac{\pi}{2 \beta} \{ 1 - \erf(\mu \beta) \},
%\]
%If we let $\beta = 2^{-1/2}$ and $x^2+1/2 = z$, the above identity reduces to \eqref{eq:erdelyi} in $\mu$. 

The second candidate is the symmetric stable distribution, defined by its characteristic function $\phi(t) = \exp( -|t|^{\alpha}), 0 < \alpha \leq 2$. It admits a normal scale mixture representation with mixing density as $f(v) = 2^{-1} s_{\alpha/2}(v/2), v > 0$, where $s_{\alpha/2}$ is the positive stable density with index $\alpha / 2$ \citep{gneiting1997normal}. The exponential power density arising as a dual of the symmetric stable density also has a normal scale mixture representation with important application in
Bayesian bridge regression \citep{polson_bayesian_2014}.
\[
e^{-|x|^\alpha} = \int_0^{\infty} e^{-x\eta} g(\eta) d\eta, \quad g(\eta) = \sum_{j=1}^{\infty} (-1)^j \frac{\eta^{-j \alpha-1}}{j! \Gamma(-\alpha j)}
  \;,
\]
\citet{polson_bayesian_2014} derive this as a limiting result of the scale-mixture of beta representation for $k$-montone densities and utilizing the complete monotonicity of exponential power density. Regularization, in this case, is an outcome of a normal scale mixture with respect to an $\alpha$-stable random variable.  We conjecture that these two results follow from the \CS{} formula \eqref{eq:identity}. Other potential applications include using Liouville formula to recognize and generate global-local mixtures, and to calculate higher-order closed-form moments $E(X^n)$ for random variables $X$ that admit a global-local representation. 

\bibliographystyle{biometrika}
\bibliography{glref}

\end{document}
