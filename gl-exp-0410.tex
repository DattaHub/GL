%\documentclass[manuscript]{biometrika}
\documentclass[lineno]{biometrika}
\usepackage{amsmath,amssymb}
\usepackage{ntheorem}

%\usepackage{amssymb}
%\usepackage{amsthm}
\usepackage{subcaption}
\usepackage{multirow}

%% Please use the following statements for
%% managing the text and math fonts for your papers:
\usepackage{times}
%\usepackage[cmbold]{mathtime}
\usepackage{bm}
\usepackage{natbib}
\usepackage[plain,noend]{algorithm2e}

%\usepackage{array}
%
% Math commands by Thomas Minka 
%
% Revised by Jyotishka Datta & Brandon Willard
% Acknowledgement: JD received this from Prof. Alan Qi.
%


%% Special shortcuts for us
\def\Polya{P{\'o}lya}
\def\CS{Cauchy-Schl\"omilch}
\def\PG{P{\'o}lya-Gamma}

%\setlength{\textfloatsep}{10pt plus 1.0pt minus 2.0pt}
%\setlength{\floatsep}{12.0pt plus 2.0pt minus 5.0pt}
%\setlength{\intextsep}{12.0pt plus 2.0pt minus 5.0pt}
%\setlength{\belowcaptionskip}{-2pt}

%\def\distrib{\mathrel{\ooalign{%
%  \raisebox{0.75\height}{{\small{ind}}}\cr\hidewidth$\sim$\hidewidth\cr}}}
%  
\newcommand{\var}{{\rm var}}
\newcommand{\Tr}{^{\rm T}}
\newcommand{\rmlog}{\rm log}
\newcommand{\vtrans}[2]{{#1}^{(#2)}}
\newcommand{\kron}{\otimes}
\newcommand{\schur}[2]{({#1} | {#2})}
\newcommand{\schurdet}[2]{\left| ({#1} | {#2}) \right|}
\newcommand{\had}{\circ}
\newcommand{\diag}{{\rm diag}}
\newcommand{\invdiag}{\diag^{-1}}
\newcommand{\rank}{{\rm rank}}
% careful: ``null'' is already a latex command
\newcommand{\nullsp}{{\rm null}}
\newcommand{\tr}{{\rm tr}}
\renewcommand{\vec}{{\rm vec}}
\newcommand{\vech}{{\rm vech}}
\renewcommand{\det}[1]{\left| #1 \right|}
\newcommand{\pdet}[1]{\left| #1 \right|_{+}}
\newcommand{\pinv}[1]{#1^{+}}
\newcommand{\erf}{{\rm erf}}
\newcommand{\hypergeom}[2]{{}_{#1}F_{#2}}

% boldface characters
\renewcommand{\a}{{\bf a}}
\renewcommand{\b}{{\bf b}}
\renewcommand{\c}{{\bf c}}
\renewcommand{\d}{{\rm d}}  % for derivatives
\newcommand{\e}{{\rm e}} % for exponentials
\newcommand{\f}{{\bf f}}
\newcommand{\g}{{\bf g}}
\newcommand{\h}{{\bf h}}
%\newcommand{\k}{{\bf k}}
% in Latex2e this must be renewcommand
\renewcommand{\k}{{\bf k}}
\newcommand{\m}{{\bf m}}
\newcommand{\n}{{\bf n}}
%\renewcommand{\o}{{\bf o}}
\newcommand{\p}{{\bf p}}
%\newcommand{\q}{{\bf q}}
\renewcommand{\r}{{\bf r}}
\newcommand{\s}{{\bf s}}
\renewcommand{\t}{{\bf t}}
\renewcommand{\u}{{\bf u}}
\renewcommand{\v}{{\bf v}}
\newcommand{\w}{{\bf w}}
%\newcommand{\x}{{\bf x}}
\newcommand{\y}{{\bf y}}
%\newcommand{\z}{{\bf z}}
\newcommand{\A}{{\bf A}}
\newcommand{\B}{{\bf B}}
%\newcommand{\C}{{\bf C}}
\newcommand{\D}{{\bf D}}
\newcommand{\E}{{\bf E}}
\newcommand{\F}{{\bf F}}
%\newcommand{\G}{{\bf G}}
\renewcommand{\H}{{\bf H}}
\newcommand{\I}{{\bf I}}
\newcommand{\J}{{\bf J}}
\newcommand{\K}{{\bf K}}
\renewcommand{\L}{{\bf L}}
\newcommand{\M}{{\bf M}}
\newcommand{\Nor}{{\cal N}}  % for normal density
%\newcommand{\N}{{\bf N}}
\renewcommand{\O}{{\bf O}}
\renewcommand{\P}{{\bf P}}
\newcommand{\Q}{{\bf Q}}
\newcommand{\R}{{\bf R}}
%\renewcommand{\S}{{\bf S}}
%\newcommand{\T}{{\bf T}}
%\newcommand{\U}{{\bf U}}
\newcommand{\V}{{\bf V}}
\newcommand{\W}{{\bf W}}
\newcommand{\X}{{\bf X}}
\newcommand{\Y}{{\bf Y}}
\newcommand{\Z}{{\bf Z}}

% this is for latex 2.09
% unfortunately, the result is slanted - use Latex2e instead
%\newcommand{\bfLambda}{\mbox{\boldmath$\Lambda$}}
% this is for Latex2e
\newcommand{\bfLambda}{\boldsymbol{\Lambda}}

% Yuan Qi's boldsymbol
\newcommand{\bsigma}{\boldsymbol{\sigma}}
\newcommand{\balpha}{\boldsymbol{\alpha}}
\newcommand{\bpsi}{\boldsymbol{\psi}}
\newcommand{\bphi}{\boldsymbol{\phi}}
\newcommand{\bbeta}{\boldsymbol{\beta}}
%\newcommand{\Beta}{\boldsymbol{\eta}}
\newcommand{\btau}{\boldsymbol{\tau}}
\newcommand{\bvarphi}{\boldsymbol{\varphi}}
\newcommand{\bzeta}{\boldsymbol{\zeta}}
\newcommand{\bnabla}{\boldsymbol{\nabla}}
\newcommand{\blambda}{\boldsymbol{\lambda}}
\newcommand{\bLambda}{\mathbf{\Lambda}}

\newcommand{\btheta}{\boldsymbol{\theta}}
\newcommand{\bpi}{\boldsymbol{\pi}}
\newcommand{\bPi}{\boldsymbol{\Pi}}
\newcommand{\bxi}{\boldsymbol{\xi}}
\newcommand{\bSigma}{\boldsymbol{\Sigma}}

\newcommand{\bgamma}{\mathbf{\gamma}}
\newcommand{\bGamma}{\mathbf{\Gamma}}

\newcommand{\bmu}{\boldsymbol{\mu}}
\newcommand{\bnu}{\boldsymbol{\nu}}
\newcommand{\bPsi}{\mathbf{\Psi}}
\newcommand{\bepsilon}{\boldsymbol{\epsilon}}
\newcommand{\bOmega}{\boldsymbol{\Omega}}

\newcommand{\1}{{\bf 1}}
\newcommand{\0}{{\bf 0}}

%\newcommand{\comment}[1]{}

\newcommand{\bs}{\backslash}
\newcommand{\ben}{\begin{enumerate}}
\newcommand{\een}{\end{enumerate}}
\newcommand{\beq}{\begin{equation}}
\newcommand{\eeq}{\end{equation}}

 \newcommand{\notS}{{\backslash S}}
 \newcommand{\nots}{{\backslash s}}
 \newcommand{\noti}{{\backslash i}}
 \newcommand{\notj}{{\backslash j}}
 \newcommand{\nott}{\backslash t}
 \newcommand{\notone}{{\backslash 1}}
 \newcommand{\nottp}{\backslash t+1}
% \newcommand{\notz}{\backslash z}

\newcommand{\notk}{{^{\backslash k}}}
%\newcommand{\noti}{{^{\backslash i}}}
\newcommand{\notij}{{^{\backslash i,j}}}
\newcommand{\notg}{{^{\backslash g}}}
\newcommand{\wnoti}{{_{\w}^{\backslash i}}}
\newcommand{\wnotg}{{_{\w}^{\backslash g}}}
\newcommand{\vnotij}{{_{\v}^{\backslash i,j}}}
\newcommand{\vnotg}{{_{\v}^{\backslash g}}}
\newcommand{\half}{\frac{1}{2}}
\newcommand{\quart}{\frac{1}{4}}
\newcommand{\msgb}{m_{t \leftarrow t+1}}
\newcommand{\msgf}{m_{t \rightarrow t+1}}
\newcommand{\msgfp}{m_{t-1 \rightarrow t}}

\newcommand{\proj}[1]{{\rm proj}\negmedspace\left[#1\right]}
\newcommand{\argmin}{\operatornamewithlimits{argmin}}
\newcommand{\argmax}{\operatornamewithlimits{argmax}}

\newcommand{\dif}{\mathrm{d}}
\newcommand{\abs}[1]{\lvert#1\rvert}
\newcommand{\norm}[1]{\lVert#1\rVert}
\newcommand{\vectornorm}[1]{\left|\left|#1\right|\right|}

\newcommand{\Norm}{\mathcal{N}}
\newcommand{\bx}{{\bf x}}
\newcommand{\ba}{{\bf a}}
\newcommand{\bb}{{\bf b}}
\newcommand{\bc}{{\bf c}}
\newcommand{\bd}{{\bf d}}
\newcommand{\bX}{{\bf X}}
\newcommand{\by}{{\bf y}}
\newcommand{\IG}{\mathcal{IG}}
\newcommand{\dd}[2]{\frac{\partial #1}{\partial #2}}
\newcommand{\lhat}[1][i]{\hat\lambda_{#1}^{-1(g)}}
\newcommand{\what}[1][j]{\hat\omega_{#1}^{-1(g)}}
\newcommand{\bone}{{\bf 1}}
\newcommand{\Li}{\hat\Lambda^{-1(g)}}
\newcommand{\Oi}{\hat\Omega^{-1(g)}}
\newcommand{\iid}{\stackrel{\mathrm{iid}}{\sim}}
\newcommand{\iidp}{\stackrel{\mathrm{P}}{=}}
\newcommand{\iidd}{\stackrel{\mathrm{D}}{=}}
\newcommand{\defeq}{\operatorname{:=}}

% the last {} is a hack for double subscript errors
\newcommand{\estHsp}{\ensuremath{{\hat{\theta}}_{HS+}}{}}
\newcommand{\estHs}{\ensuremath{{\hat{\theta}}_{HS}}{}}
\newcommand{\estJs}{\ensuremath{{\hat{\theta}}_{JS}}{}}
\newcommand{\MSE}{\mathrm{MSE}}
%
% Meijer-G additions
%
%\DeclarePairedDelimiterX\MeijerM[3]{\lparen}{\rparen}%
%{\begin{smallmatrix}#1 \\ #2\end{smallmatrix}\delimsize\vert\,#3}
%
%\newcommand\MeijerG[8][]{%
%  G^{\,#2,#3}_{#4,#5}\MeijerM[#1]{#6}{#7}{#8}}
%
%\WithSuffix\newcommand\MeijerG*[7]{G^{\,#1,#2}_{#3,#4}\MeijerM*{#5}{#6}{#7}}
%% end Meijer-G
%
%%
%% Generalized Hypergeometric Function (pFq)
%%
%\DeclarePairedDelimiterX\pFqM[3]{\lparen}{\rparen}%
%{\begin{smallmatrix}#1 \\ #2\end{smallmatrix}\delimsize\vert\,#3}
%
%\newcommand\pFq[6][]{%
%  {}_{#2}F_{#3}\pFqM[#1]{#4}{#5}{#6}}

%\WithSuffix\newcommand\pFq*[5]{{}_{#1}F_{#2}\pFqM*{#3}{#4}{#5}}
% end pFq

%\newtheorem{theorem}{Theorem}
%\numberwithin{theorem}{section}
%\newtheorem{Proof}{Proof}
%\newtheorem{Def}{Definition}
%\numberwithin{Def}{section}
%\newtheorem{remark}{Remark}
%\numberwithin{remark}{section}
%\newtheorem{Qes}{Question}
%\newtheorem{proposition}{Proposition}
%\numberwithin{proposition}{section}
%\newtheorem{lemma}{Lemma}
%\numberwithin{lemma}{section}
%\newtheorem{Cor}{Corollary}
%\numberwithin{Cor}{section}
%\newtheorem{Exa}{Example}
%\newtheorem{Eq}{Equation}
%\newtheorem{assn}{Assumption}
%\newtheorem{result}[theorem]{Result}


%\usepackage{booktabs,array}
%\def\Midrule{\midrule[\heavyrulewidth]}
%\newcount\rowc
%
%\makeatletter
%\def\ttabular{%
%\hbox\bgroup
%\let\\\cr
%\def\rulea{\ifnum\rowc=\@ne \hrule height 1.0pt \fi}
%\def\ruleb{
%\ifnum\rowc=1\hrule height 1.0pt  \else
%\ifnum\rowc=3\hrule height 0.0pt%\heavyrulewidth 
%\ifnum\rowc= 3  \hrule height 0.5pt \else%\heavyrulewidth 
%\ifnum\rowc= 5  \hrule height 0.5pt \else%\heavyrulewidth 
%\ifnum\rowc= 7  \hrule height 0.5pt \else%\heavyrulewidth 
%\ifnum\rowc= 9  \hrule height 0.5pt \else%\heavyrulewidth 
%\ifnum\rowc= 11  \hrule height 0.5pt %\heavyrulewidth 
  %\else \hrule height 0pt%\lightrulewidth
%\fi\fi\fi\fi\fi\fi}
%\valign\bgroup
%\global\rowc\@ne
%\rulea
%\hbox to 7em{\strut \hfill##\hfill}%
%\ruleb
%&&%
%\global\advance\rowc\@ne
%\hbox to 7em{\strut\hfill##\hfill}%
%\ruleb
%\cr}
%\def\endttabular{%
%\crcr\egroup\egroup}



\begin{document}

\jname{Biometrika}
%% The year, volume, and number are determined on publication
\jyear{2016}
\jvol{}
\jnum{}
%% The \doi{...} and \accessdate commands are used by the production team
%\doi{10.1093/biomet/asm023}
\accessdate{Advance Access publication on 31 July 2016}
\copyrightinfo{\Copyright\ 2016 Biometrika Trust \goodbreak {\em Printed in Great Britain}}

%% These dates are usually set by the production team
\received{April 2016}
\revised{September 2016}

%% The left and right page headers are defined here:
\markboth{Bhadra, Datta, Polson, and Willard}{Global-Local Mixtures}

%% Here are the title, author names and addresses
\title{Global-Local Mixtures}

\author{Anindya Bhadra}
\affil{Department of Statistics, Purdue University. \email{bhadra@purdue.edu.}}
\author{Jyotishka Datta}
\affil{Department of Statistical Science, Duke University. \email{jd298@stat.duke.edu}}
\author{Nicholas G. Polson \and Brandon Willard }
\affil{The University of Chicago Booth School of Business. 
  \email{ngp@chicagobooth.edu \and bwillard@uchicago.edu}}

\maketitle
\begin{abstract}
\noindent 
We show how global-local mixtures can be generated using integral identities known as the Schl\"omilch and the Liouville transformation. These identities lead to simpler proofs of many known normal-scale mixture distributions such as the Laplace or Lasso, the logit and the quantile. We also show how they lead to new probability distributions as global-local scale mixtures of appropriate baseline densities. 
\end{abstract}

\begin{keywords}
Global-local mixture, Scale mixture, Stable laws.
\end{keywords}
\section{Introduction}
%\section{Global-Local Mixtures}
Many statistical problems involve regularization penalties, that can be viewed as global-local mixture prior distributions \citep{polson2011data, hans2011comment}. A random variable $X = (x_1, \ldots, x_p)^{\T}$ is said to have a global-local mixture distribution if we can write its marginal probability density function as: 
\begin{align*}
p(x_i) & = \int_{0}^{\infty} p(x_i \mid \tau) p(\tau) d\tau \text{ where, }\\
p( x_i \mid \tau) & = \int_{0}^{\infty} p(x_i \mid \lambda_i, \tau) p(\lambda_i \mid \tau) p(\tau) d\lambda_i
\end{align*}
Here $\tau$ is a global mixing parameter and $\lambda_i$'s are the local ones. There is great interest in being able to analytically calculate the marginal distributions of a global-local mixture - primarily as a log-penalty in sparse Bayesian estimation. Our approach here is to show how two simple integral identities, Schl\"omilch and Liouville integrals, derive useful `closed-form' identities very quickly. Moreover. it provides a framework for generating new random variables via the transformation $Y = a X - b X^{-1}$ for hyper-parameters $a$ and $b$. It also provides a transformation framework that links together different global-local mixtures. 

We start by stating the two key integral identities that we shall use throughout the paper, the Cauchy-Schl\"omilch transformation: 
\begin{equation}
\int_0^\infty f \left\{ ( a x - b x^{-1} )^2 \right\} d x = \frac{1}{2a} \int_0^\infty f(y^2) d y \label{eq:identity}
\end{equation}
and the Liouville transformation:
\begin{equation}
\int_{0}^{\infty} f\left(ax + \frac{b}{x} \right) \frac{dx}{\sqrt{x}} = \frac{1}{\sqrt{a}} \int_{0}^{\infty} f \left( 2\sqrt{ab} + y \right) \frac{dy}{\sqrt{y}} \label{eq:liouville}
\end{equation}
See \citep{boros2006irresistible,baker2008probabilistic,jones2014generating} for a detailed discussion of these identities. 

\begin{proof}
Equation \eqref{eq:identity} follows from the simple transformation $t = b/ax$: 
\begin{equation*}
  I = \int_{0}^{\infty} f \left\{(ax - b/x)^2 \right\} dx = \int_{0}^{\infty} f \left\{(at - b/t)^2 \right\} \frac{b}{at^2}.
\end{equation*}
Now, adding we get: 
\begin{align*}
2 I & = \int_{0}^{\infty} f \left\{(at - b/t)^2 \right\} \left( 1+\frac{b}{at^2} \right) dt.
\end{align*}
Then, transforming $y = b/t - at$ that implies $dy = -a (1+b/at^2)$, we get \eqref{eq:identity}, namely, $I = (2a)^{-1} \int_{0}^{\infty} f(y^2) dy$. 
\end{proof}
The proof for the Liouville transformation follows a very similar proof which we omit for the sake of brevity. 

A useful generalization of the Cauchy-Schl\"omilch transformation is: 
\begin{equation}
  \int_0^\infty f \left[ \{x-g(x)\}^2 \right] dx = \int_0^\infty f( y^2 ) dy \label{eq:gen}
\end{equation}
Where $g(x)=g^{-1}(x)$ is a self-inverse function such as $g(x) = b/x$ or $g(x) = -a^{-1}\log\{1-\exp(a x)\}$, see \citep{jones2010distributions,jones2014generating} for a detailed discussion. %The proof is along the same lines and we skip it for the sake of brevity. 

\section{Global Local Scale Mixture}
\subsection{Lasso as a Normal Scale Mixture}
It is well known that the double exponential or Laplace distribution arises as a normal scale mixtures with a Gamma mixing density \citep{andrews1974scale}. A simple transformation proof follows from the \CS formula where, $ f(x) = e^{-x} $ in \eqref{eq:identity}. The normal integral identity, $ \int_{0}^{\infty} f(y^2) dy = \int_0^\infty e^{-y^2} d y = \sqrt{\pi}/2 $ becomes $\int_{0}^{\infty} f(ax-bx^{-1}) dx$:  
$$
\int_0^\infty e^{- a x^2 - b / x^2 } d x = \frac{1}{2} \sqrt{ \frac{\pi}{a}} e^{- 2 \sqrt{ab}}
$$
This can also be written with $ x = b^{1/2} t $ and $c = ab$ to get the \citet{andrews1974scale} result for double exponential: 
\begin{equation}
  \int_0^\infty \exp \{- c t^2 - 1/t^2 \} d t = \half (\pi/c)^{1/2} \exp (-2 \sqrt{c}) \label{eq:andrews}
\end{equation}
The advantage of this short proof is that the Laplace density can be viewed as a transformation of the normal. 

\begin{remark}
(Need to check this carefully!) 

The usual identity for lasso is the following proved by \citet{levy1940certains}.
$$
\int_{0}^{\infty} \frac{a}{\sqrt{2\pi x^3}} e^{-a^2/2x} e^{-\lambda x} dx = e^{-a\sqrt{2\lambda}} 
$$
%\end{remark}
%Must apply to other exponential family mixtures? $ f(x ) = x^{\alpha-1} e^{-x} $ etc.
%\begin{remark} 
We can also write the above identity as: 
$$
E \left[\exp\{-\lambda^2/4\gamma_{\half,\half} \} \right] = \exp(-\lambda), 
$$ where $\gamma_{\half, \half}$ is a Gamma random variable. 
\end{remark} 

\subsection{Logit and Quantile as Normal Scale Mixtures}
Bayesian approaches to logistic modeling involves the \PG distribution \citep{polson2013bayesian}. The two key integral identities for the hyperbolic-GIG \citep{barndorff1977infinite} and Z-\Polya mixtures \citep{polson2013bayesian,barndorff1982normal} are 
\begin{align}
\frac{\alpha^2-\kappa^2}{2\alpha} e^{-\alpha|\theta-\mu| + \kappa(\theta-\mu)} & = \int_0^{\infty} \phi(\mu + \kappa \lambda, \lambda) p_{GIG}(1,0,\sqrt{\alpha^2-\kappa^2}) (\lambda) d\lambda \label{eq:GIG}\\
\frac{1}{B(\alpha,\kappa)} \frac{e^{\alpha(\theta-\mu)}}{(1+e^{\theta-\mu})^{\alpha + \kappa}} & = \int_0^{\infty} \phi(\mu + \kappa \lambda, \lambda) p_{Polya}(\alpha,\kappa) (\lambda) d\lambda \label{eq:polya}
\end{align}
where $\phi(\mu + \kappa \lambda, \lambda)$ denotes the normal density. The corresponding mixture distributions are Generalized Inverse-Gaussian and the Polya-Z denoted by $p_{GIG}$ and $p_{Polya}$ respectively. Rather surprisingly, the logit and quantile identities \eqref{eq:GIG} and \eqref{eq:polya} can also be derived from the integral identity \eqref{eq:identity}. We give the proofs below: 

\begin{lemma}
The integral identities \eqref{eq:GIG} and \eqref{eq:polya} admit a global-local mixture representation via the \CS transformation in \eqref{eq:identity}. 
\end{lemma}
\begin{proof}
%\noindent \textbf{Proof of \eqref{eq:GIG}} \\
To prove \eqref{eq:GIG}, let $f(x) = e^{-x^2/2}$, $a = \alpha, b = |\theta-\phi|$ in \eqref{eq:identity}, then we have: 
$$
\sqrt{\frac{2}{\pi}} \int_{0}^{\infty} e^{-\half \left(\alpha y - \frac{|\theta-\mu|}{y} \right)^2} dy = \frac{1}{\alpha}\frac{1}{\sqrt{2\pi}} \int_0^{\infty} e^{-\half y^2} dy = \frac{1}{\alpha}
$$
Letting $\nu = y^2$ and re-arranging the constant terms, we get the relation:
$$
\frac{1}{\alpha} e^{-\alpha|\theta-\mu|} = \int_{0}^{\infty} e^{-\left( \frac{(\theta-\mu)^2}{2\nu} + \frac{\alpha^2}{2} \nu \right)} \frac{1}{\sqrt{2\pi\nu}} d\nu.
$$
Now, multiplying both sides by $2^{-1}(\alpha^2-\kappa^2) \exp\{\kappa(\theta-\mu)\}$, re-arranging terms to complete squares, so that: 
\begin{equation*}
  \frac{\alpha^2-\kappa^2}{2\alpha} e^{-\alpha|\theta-\mu| + \kappa(\theta-\mu)} = \int_0^{\infty} \phi(\theta | \mu + \kappa \nu, \nu) \frac{\alpha^2-\kappa^2}{2} e^{-\frac{\alpha^2-\kappa^2}{2} \nu} d \nu.
\end{equation*}
The mixing distribution on the right hand side is the exponential distribution with rate parameter $(\alpha^2-\kappa^2)/2$, which is a special case of the Generalized inverse Gaussian or GIG distribution. Etienne Halphen circa 1941 \citep{seshadri1997halphen} introduced the GIG and \citet{barndorff1977infinite} studied its properties. The density with parameters $(\lambda, \delta, \gamma)$ has the form: 
$$
p_{GIG}(x \mid \lambda, \delta, \gamma) = \frac{(\gamma/\delta)^{\lambda}}{2K_{\lambda}(\delta \gamma)} x^{\lambda-1} \exp \left\{ -\half (\delta^2 x^{-1} + 
\gamma^2 x )  \right\}; \; x > 0
$$
where $K_{\lambda}$ is the modified Bessel function of the second kind. When $\delta$ or $\gamma$ is zero (in our case, $\delta = 0$), the normalizing constant is to be replaced by the corresponding limits as $K_{\lambda}(u) \sim \Gamma(|\lambda|)2^{|\lambda|-1} u^{|\lambda|}$ when $\lambda \neq 0$. It turns out that if $\delta=0$, GIG is identical to a Gamma distribution with density 
$$
p_{GIG}(x \mid \lambda, \delta = 0 , \gamma) = \frac{(\alpha)^{\lambda}}{\Gamma(\lambda)} x^{\lambda-1} \exp\{ -\alpha x \}; \; x > 0, \alpha = \gamma^2 / 2 
$$
%\end{proof} 
%
%%%%%%%%%%%
%\begin{proof}
%\noindent \textbf{Proof of \eqref{eq:polya}} \\
Next we present a simple proof for the \PG mixture in \eqref{eq:polya}. First, write $\kappa$ for $a-b/2$: 
\begin{equation}
  \frac{(e^{\psi})^a}{(1+e^{\psi})^b} = 2^{-b} e^{\kappa \omega} \int_0^{\infty} e^{-\omega \psi^2/2} p(\omega) d\omega \label{eq:pg}
\end{equation}
where $\omega \sim PG(b,0)$, the\PG distribution with density given by: 
$$
p(x \mid b, 0) = \frac{2^{(b-1)}}{\Gamma(b)} \sum_{n=0}^{\infty} (-1)^n \frac{\Gamma(n+b)}{\Gamma(n+1)} \frac{2n+b}{\sqrt{2\pi x^3}} e^{-\frac{(2n+b)^2}{8x}} 
$$
The logit function corresponds to $a=0,b=1$ in \eqref{eq:pg}. The identity then simplifies to:  
\begin{equation}
  \frac{1}{1+e^{\psi}} = \half e^{-\half \psi} \int_0^{\infty} e^{-\frac{\psi^2}{2} \omega} p(\omega) d\omega 
  \text{ where } p(\omega) = \sum_{n=0}^{\infty} (-1)^n \frac{2n+1}{\sqrt{2\pi \omega^3}} e^{-\frac{(2n+1)^2}{8\omega}}
  \label{eq:logit}
\end{equation}
We will now prove that the right hand side equals the left hand side in \eqref{eq:logit}. To do this, write the left-hand side as:
\begin{align*}
I & = \half e^{-\psi/2} \int_0^{\infty} e^{-\frac{\psi^2}{2} \omega} \sum_{n=0}^{\infty} (-1)^n \frac{2n+1}{\sqrt{2\pi \omega^3}} e^{-\frac{(2n+1)^2}{8\omega}}d \omega \\
& = \half e^{-\psi/2}  \sum_{n=0}^{\infty} (-1)^n \frac{2n+1}{\sqrt{2\pi}} \int_0^{\infty} e^{- \left( \frac{\psi^2}{2}  \omega + \frac{(2n+1)^2}{8\omega} \right) } \frac{1}{\sqrt{ \omega^3}} d\omega 
\end{align*}
Now, using the change of variable $\omega = t^{-2}$, we get 
\begin{align*}
I & = e^{-\psi/2}  \sum_{n=0}^{\infty} (-1)^n \frac{(2n+1)}{\sqrt{2\pi}} \int_{0}^{\infty} e^{-\left( \frac{\psi^2}{2t^2} + \frac{(2n+1)^2 t^2}{8} \right)} d t \\
& =  e^{-\psi/2}  \sum_{n=0}^{\infty} (-1)^n \frac{(2n+1)}{\sqrt{2\pi}} \int_{0}^{\infty} e^{-\half \left( \frac{(2n+1)t}{2} - \frac{\psi}{t}\right)^2 } e^{- \frac{(2n+1)}{2} \psi}d t \\
& = \sum_{n=0}^{\infty} (-1)^n \left[  e^{-(n+1)\psi} \frac{(2n+1)}{\sqrt{2\pi}} \left\{ \int_{0}^{\infty}e^{-\half \left( \frac{(2n+1)t}{2} - \frac{\psi}{t}\right)^2 } dt \right\} \right].
\end{align*}
Finally, apply our first integral identity to the integral within curly brackets to obtain: 
$$ 
\int_{0}^{\infty}e^{-\half \left( \frac{(2n+1)t}{2} - \frac{\psi}{t}\right)^2 } dt = \int_0^{\infty} e^{-\half y^2} dy \frac{1}{2n+1} = \frac{\sqrt{2\pi}}{2n+1}
$$
Putting things back together, we arrive at the logit function: 
$$
I = \sum_{n=0}^{\infty} (-1)^n e^{-(n+1)\psi} = \frac{1}{1+e^{\psi}}.
$$
\end{proof}

An alternative proof using Laplace transformation is in \cite{polson2013bayesian}. 
\begin{remark}
In the special case $\alpha = \kappa$, Equation \eqref{eq:GIG} and the limiting result yields the following identity: 
$$
(\alpha^2-\kappa^2)^{-1}p_{GIG}(1,0,\sqrt{\alpha^2-\kappa^2}) \equiv 1
$$
Hence it is as if the latent variable has a marginal improper uniform prior $p(\lambda) = 1, \forall \lambda \in \Re$. We can also write the result as: 
\begin{equation}
  \int_{0}^{\infty} \phi(b \mid -a\lambda, c\lambda) d\lambda = a^{-1} \exp(-2 \max(ab/c,0)) \label{eq:svm}
\end{equation}
This is useful in support vector machines \cite{polson2011data}.
\end{remark}

\begin{remark}
In a discussion of \citet{polson2011data}, \citet{hans2011comment} pointed out that the elastic-net regression can be recast as a normal mean variance mixture problem where the mixing density belongs to a `orthant-normal' family of distributions. The orthant-normal prior on a single regression coefficient $\beta$ given hyper-parameters $\lambda_1$ and $\lambda_2$ is given by: 
\[
p (\beta \mid \lambda_1, \lambda_2)  = 
  \begin{cases} 
   \phi(\beta \mid \frac{\lambda_1}{2\lambda_2}, \frac{\sigma^2}{\lambda_2}) / 2\Phi(\frac{-\lambda_1}{2\sigma\sqrt{\lambda_2}}) & \text{if } \beta < 0 \\
   \phi(\beta \mid \frac{-\lambda_1}{2\lambda_2}, \frac{\sigma^2}{\lambda_2}) / 2\Phi(\frac{-\lambda_1}{2\sigma\sqrt{\lambda_2}})       & \text{if } \beta \geq 0
  \end{cases} \label{eq:hans}
\]
Then \citet{hans2011comment} proves the following theorem: 
\begin{theorem*}
Under the orthant normal prior on $\beta$ given above and prior on $\lambda_2$ with the density
$$
p(\lambda_2 \mid \lambda_1, \sigma^2) = \frac{\lambda_1^2}{2\sigma^2\lambda_2^2} \Phi \left( \frac{-\lambda_1}{2\sigma \sqrt{\lambda_2}} \right), \lambda_2 > 0 
$$
the induced marginal prior on $\beta$ is a double-exponential, in particular, $\beta \mid \lambda_1, \sigma^2 \sim DE(\lambda_1/\sigma^2)$.
\end{theorem*}
This result comes as a corollary of \eqref{eq:svm}. 

%the limiting result stated in the last remark, i.e. it follows from the identity 
%$$
%\int_{0}^{\infty} \phi(b \mid -a\lambda, c\lambda) d\lambda = a^{-1} exp(-2 \max(ab/c,0)).
%$$
\end{remark}
%\subsection{Stable Laws}
%The symmetric stable distribution is defined by its characteristic function $\phi(t) = \exp\{ -|t|^{\alpha} \}$, with exponent $\alpha \in (0,2]$. The symmetric stable distribution also admits a normal scale mixture representation as where the mixing density is given by: 
%$$
%f(v) = \half s_{\alpha/2} \left( \frac{v}{2} \right), v > 0 
%$$
%where $s_{\alpha/2}$ is the density of the positive stable distribution with index $\alpha / 2$ (Feller, 1971). This result can also shown to be a consequence of the integral identity \eqref{eq:identity}. 
%\textit{I am still working on the proof of it.}

\section{Transformations of Scale Distribution} 

\cite{jones2014generating} characterizes the densities of the form $f(x) = 2g\{ t(x) \}$ and provided conditions for $t(\cdot)$ for $f(\cdot)$ to be a density. This result allows us to generate new distributions starting from a baseline `simple' distribution and transforming the scale by a suitable function. The main result is quoted below: 
\begin{proposition}\label{prop:jones}
Let $\Pi: \mathcal{D} \to \mathcal{S}_{f}$ be a piecewise differentiable monotone increasing function with inverse $t$, where $\mathcal{D} \supset \mathcal{S}_{g} \ni 0$. Suppose that 
$$
\Pi(y) - \Pi(-y) = y, \text{ for all } y \in \mathcal{D}
$$
Then if $g(x)$ is a density on $\mathcal{S}_{g}$ symmetric about zero, $f(x) = 2g \{ t(x) \} \equiv 2g \{ \Pi^{-1}(x) \}$ is a density on 
$\mathcal{S}_{f}$.
\end{proposition}
Curiously, an equivalent formulation of $t(\cdot)$ is that $t(x)$ should have the form $x-s(x)$ where $s : \Re^+ \to \Re^+$ is an onto monotone decreasing function that is self-inverse, i.e. $s\{ s(x)\} = x$. This together with the \CS transformation \eqref{eq:gen} guarantees that the left hand side is a density on the appropriate domain, and provides for a way to represent the resulting $g\{t(x)\}$ as a global-local scale mixture! 
 
\cite{jones2014generating} further points out that only a few choices of $\pi(\cdot)$ leads to fully tractable formulae for its integral $\Pi(y) = \int_{-\infty}^{y} \pi(\omega) d\omega$ and its inverse $t = \Pi^{-1}$. Two special choices are the $t$ distribution with 2 degrees of freedom and the logistic. 
\begin{align}
t(2): \Pi_{T}(y) = (1/2)(y+\sqrt{4b+y^2}) & \Rightarrow \Pi_T^{-1}(x) = t_T(x) = x - (b/x) \\
Logistic: \Pi_{L}(y) = a^{-1} \log(1+e^{ay}) & \Rightarrow \Pi_L^{-1}(x) = t_L(x) = a^{-1} \log(e^{ax}-1) 
\end{align}
Now, we can use these results to generate new probability distributions with different choices of `simple' baseline function $g(\cdot)$ and derive new scale mixture representations. 

%\begin{remark} 
\subsection{Symmetric and R-symmetric distributions}

For absolutely continuous random variables supported on the positive half-line $\Re^{+}$, the density function $f(\cdot)$ has reciprocal symmetry (R-symmetry) if $f(\theta y) = f(\theta / y)$ for all $y > 0$ and some $\theta >0$ (Mudholkar and Wang, 2007). It follows from the integral identity \eqref{eq:identity} that if $f(x), x \geq 0$ is a density function, so is $g(x) = 2a f(|ax-b/x|), x >0$, and $f(\cdot)$ and $g(\cdot)$ are called the mother and the daughter pdf. As \cite{chaubey2010reciprocal} pointed out, there is an one-to-one correspondence between $f$ and $g$, and furthermore if $f(x)$ is the pdf of a symmetric real-valued random variable $Y$, the daughter pdf $g(x) = f(x-1/x), x>0$ is an R-symmetric density, and vice-versa, there exists a symmetric density $f(x) = g(x+\sqrt{1+x^2})$ for every R-symmetric density $g(x)$. Furthermore, $f(\cdot)$ is unimodal if and only if $g(\cdot)$ is unimodal. \cite{chaubey2010reciprocal} provides a few examples of generating R-symmetric densities $g$ starting from well-known symmetric densities $f$. The most well-known example of this duality is perhaps the normal density as $f$ that gives rise to the root reciprocal inverse Gaussian, abbreviated as RRIG, distribution, with density given by: 
$$
g(x) = \sqrt{\frac{2\lambda}{\pi}} \exp \left\{ - \frac{\lambda}{2} \left( x - \frac{1}{x} \right)^2 \right\}, x >0.
$$

\section{Discussion}

Apart from providing simple proofs for well-known normal scale mixture representations, the main motivation behind studying properties of the integral identities is twofold: we can generate new distributions by a suitable scale transformation $f(x) = 2g\{ t(x) \}$ of a `simple' baseline function $f$ under some conditions, and for $t(x) = x - 1/x$, we get R-symmetric densities on $\Re^+$ from well-known symmetric densities and vice-versa, and given a density $f(x)$ we can create a new global-local scale family $f(ax-bx^{-1})$, by reallocating its probability mass. 

Khintchine's theorem for unimodality of univariate distributions provides a useful tool for generating univariate and multivariate random variables \citep{bryson1982constructing}. Khintchine's theorem states that any random variable $X$ with a mode at zero can be written as a product $X = ZU$, where $U \sim U(0,1)$ and $Z$ has the density function $f_Z(z) = -zf_{X}'(z)$. \citep{bryson1982constructing} and successively \cite{jones2010distributions,jones2012khintchine} discuss how Khintchine's theorem allows us to construct both univariate and multivariate distributions, the latter with special dependence structure. \cite{jones2014generating} develops an extended Khintchine's theorem that further lets one generate random variable with unimodal density of the form $2g(t(x))$ discussed before in Proposition \ref{prop:jones}. The \CS transformation not only guarantees an `astonishingly simple' normalizing constant for $f(\cdot)$, it also establishes the wide class of unimodal densities as global-local scale mixtures. As we have discussed before, the global-local scale mixtures with conditional Gaussianity hold a special place in Statstical literature, as these models can be rapidly fit using an expectation-maximization algorithm as pointed out by \cite{polson2013data}. This has been extended in a $2011$ unpublished note by Palmer, Kreutz-Delgado and Makeig who have provided a tool for modeling multivariate dependence by writing general non-Gaussian multivariate densities as multivariate Gaussian scale mixtures. Our future goal is to extend the \CS transformation to express the wide multivariate Gaussian scale mixture models as global-local mixtures that also facilitate easy computation. \par

We end this note with conjectures that two remarkable identities arise as corollaries of the \CS transformation. The first one is a recent result by \cite{zhang2014uniform} that proves the uniform correlation mixture of the bivariate Gaussian density with unit variance is a function of the maximum norm: 
\begin{equation}
  \int_{-1}^{1} \half \frac{1}{2\pi\sqrt{1-\rho^2}} \exp \left\{ - \frac{x_1^2 + x_2^2 - 2 \rho x_1 x_2}{2(1-\rho^2)} \right\} d \rho = 
  \half (1- \Phi(\vectornorm{x}_{\infty})) \label{eq:bivar}, 
\end{equation}
where $\Phi(\cdot)$ is the standard normal distribution function and $\vectornorm{x}_{\infty} = \max \{ x_1, x_2\}$. The bivariate density on the right side of \eqref{eq:bivar} was introduced before in \cite{bryson1982constructing} as uniform mixtures of a $\chi_3$ random variate, but the representation as a uniform correlation mixture is a surprising new find. The second candidate is the symmetric stable distribution, defined by its characteristic function $\phi(t) = \exp\{ -|t|^{\alpha} \}; \alpha \in (0,2]$, also admits a normal scale mixture representation as where the mixing density is: 
$$
f(v) = \half s_{\alpha/2} \left( \frac{v}{2} \right), v > 0 
$$
where $s_{\alpha/2}$ is the density of the positive stable distribution with index $\alpha / 2$ (Feller, 1971). We conjecture that these two results follow as upshots of the \CS formula \eqref{eq:identity}. 

%%%%%%%
%\end{remark}
\bibliographystyle{biometrika}
\bibliography{glref}
\end{document}
